%!TEX program = lualatex
\documentclass[10pt,mathserif]{beamer}

\usepackage{luatexja}
\usepackage{luatexja-fontspec}
\setmainjfont[
    RawFeature={instance=Regular},
    BoldFont=Noto Sans CJK SC,
    BoldFeatures={RawFeature={instance=Bold}},
]{Noto Sans CJK SC}

\setmonofont[
    RawFeature={instance=Regular},
    BoldFont=Noto Sans Mono,
    BoldFeatures={RawFeature={instance=Bold}},
    ItalicFont=Noto Sans Mono,
    ItalicFeatures={RawFeature={instance=Italic}},
]{Noto Sans Mono}

\definecolor{xdublue}{RGB}{0,65,130}

\usepackage{listings}
\lstset{
basicstyle=\small\ttfamily,
keywordstyle=\color{xdublue},
numbers=left,
numberstyle=\tiny,
frame=leftline,
tabsize=4,
inputpath=exercise
}

\lstdefinestyle{term}
{basicstyle=\ttfamily, numbers=none, frame=single, breaklines=true,
moredelim={[is][keywordstyle]{@@}{@@}}}

\newcommand{\lstcode}[1] { \lstinputlisting[language=C++]{code/#1} }
\newcommand{\lstterm}[1] { \lstinputlisting[style=term]{code/#1} }

\usepackage{ulem}

\usetheme[xdblue]{XDUstyle}

\title{Competitive Programming 101}
\subtitle{习题讲解}
\institute{西安电子科技大学程序设计竞赛实训基地}
\author{席若尧}
\date{2021 年 7 月 21 日}
	
\begin{document}%
{\xdbg \frame[plain,noframenumbering]{\titlepage}}

\newcommand{\paragraph}[1]{}

\begin{frame}{第 1 题}
	\paragraph{第 1 题} 下列选项中,ICPC 现场赛提供的是:

A. Vim \quad
B. MATLAB \quad
C. Dev-C++ \quad
D. GCC \quad

	\pause
	\begin{itemize}
		\item 选 AD
	\end{itemize}
\end{frame}

\begin{frame}{第 2 题}
	\paragraph{第 2 题} 下列功能中,DOMJudge 支持的是:

A. 提交代码 \quad
B. 打印代码 \quad
C. 查看编译错误信息 \quad
D. 向裁判提问 \quad

	\pause
	\begin{itemize}
		\item 选 ABCD
	\end{itemize}
\end{frame}

\begin{frame}{第 3 题}
	\paragraph{第 3 题} 已知程序

\lstinputlisting[language=c++]{p3.cc}

下列选项中,可能是该程序输出的是:

A. 0 1 \quad
B. 0 0 \quad
C. 1 0 \quad
D. 114514 1919810 \quad

\end{frame}

\begin{frame}{第 3 题}
	\begin{itemize}
		\item 选 AC
		\item 就是上课讲的 \lstinline{Point p(read(), read());} 的翻版
	\end{itemize}
\end{frame}

\begin{frame}{第 4 题}
	\input{exercise/p4.tex}
	\pause
	\begin{itemize}
		\item 选 ABCD
		\item 这是一个未定义行为,所以什么都可能发生
	\end{itemize}
\end{frame}

\begin{frame}{第 5 题}
	\paragraph{第 5 题} 下列选项中,属于 C++ 语法错误的是:
\begin{itemize}
	\item A. 没定义 \lstinline{main} 函数
	\item B. 试图定义数组 \lstinline{int a[-1];}
	\item C. 将 \lstinline{if (x)} 打成 \lstinline{if x}
	\item D. 将 \lstinline{ios::sync_with_stdio();} 打成 \lstinline{ios::sync_with_stdio;}
\end{itemize}

	\pause
	选 C
	\begin{itemize}
		\item B 虽然是编译错误,但并不违反语法
		\item A 完全不违反语法,而且严格来说是未定义行为
		\item D 甚至都不算“错误”……
	\end{itemize}
\end{frame}

\begin{frame}{第 6 题}
	\paragraph{第 6 题} 你提交代码前,先用极限数据本地测试程序,
发现输出了奇怪的负数。那么,下列选项中,适合立刻进行的是:
\begin{itemize}
	\item A. 在程序中加入输出中间结果的语句,看看哪里开始出现奇怪的负数
	\item B. 打开 \lstinline{-Wconversion} 看一下有没有警告
	\item C. 打开 \lstinline{-fsanitize=undefined} 看一下有没有溢出
	\item D. 直接交上去,WA 了再说
\end{itemize}

	\pause
	选 BC
	\begin{itemize}
		\item D 不会有人选罢……
		\item 只选 C 是不够的,因为 \lstinline{int a = 1ll * b * c;}
			这样的东西并不是未定义行为,但是也会导致你 WA
		\item A 需要较大的工作量,而且有时候 (比如没初始化变量)
			会出现加了输出以后就无法重现问题的现象
	\end{itemize}
\end{frame}

\begin{frame}{第 7 题}
	\paragraph{第 7 题} 在对拍时,使用较小的数据范围的好处是
\underline{\quad~~~~~~~~~~~~~~~~~~} 和
\underline{\quad~~~~~~~~~~~~~~~~~~}。

	\pause
	以下三项写两个就行:
	\begin{itemize}
		\item 暴力可以跑出来
		\item 方便后续调试
		\item 容易触发边界条件
	\end{itemize}
\end{frame}

\begin{frame}{第 8 题}
	\paragraph{第 8 题} \lstinline{std::sort} 要求比较运算符是
\underline{\quad~~~~~~~~~~~~~~~~~~} 关系,否则会发生未定义行为;
进一步地,如果希望排序结果唯一确定,则要求比较运算符是
\underline{\quad~~~~~~~~~~~~~~~~~~} 关系。

	\pause
	\begin{itemize}
		\item 严格弱序 (因为上课讲错了,所以写严格偏序也算对)
		\item 严格全序
	\end{itemize}
\end{frame}

\begin{frame}{第 9 题}
	\paragraph{第 9 题} (3 分) 阅读程序写结果,并解释为何会出现此结果。
(假设 C++ 实现除了服从 C++ 标准外,还服从 IEEE-754 标准。)

\lstinputlisting[language=c++]{p9.cc}

	\pause
	\begin{itemize}
		\item 输出 0
		\item 0.0 / 0.0 在 IEEE-754 下是 NaN
		\item NaN 与所有 \lstinline{double} 值用小于号比较都是
			\lstinline{false},故可以认为它与所有值等价,
			\lstinline{std::set<T>::erase} 要求删除所有等价值
		\item NaN 的比较不满足严格弱序,
			但是严格弱序只要对 \lstinline{std::set<T>}
			内部元素成立即可,因此不是 UB
	\end{itemize}
\end{frame}

\begin{frame}{第 10 题}
	\paragraph{第 10 题} (2 分) 给出一道不写 \lstinline{cin.tie(0)}
就用 \lstinline{cin}/\lstinline{cout} 输入输出会超时的题目,并解释原因。

\quad \\[1cm]

	\pause
	\begin{itemize}
		\item 例:A + B,$10^7$ 组数据
		\item “求 $10^7$ 个数之和” 是不行的……
		\item “倒序输出 $10^7$ 个数” 也是不行的……
	\end{itemize}
\end{frame}

\begin{frame}{第 11 题}
	\paragraph{第 11 题} (5 分) 在长度为 $n$ 的字符串上,
使用暴力字符串匹配算法查找长度为 $m$ 的模式串第一次出现的位置。
设字符集大小为常数,求期望时间复杂度和最坏情况时间复杂度,
用 Big-$\Theta$ 表示法给出答案。要求写出计算过程。

	\pause
	\begin{itemize}
		\item 最坏:$T(n, m) = \Theta((n - m + 1)m)$
		\item 严格来说这个东西不能写成 $\Theta(nm)$
			(写成 $\Theta(n^2)$ 倒是能说通),
			但是阅卷比较松……
		\item 期望:$$T(n, m) = \sum_{k=0}^{n - m} \sum_{j=0}^{m - 1}
			|\Sigma|^{-j} = (n - m + 1)
			\frac{1 - |\Sigma|^{-m}}{1 - |\Sigma|^{-1}} =
			\Theta(n - m + 1)$$
	\end{itemize}
\end{frame}

\end{document}
